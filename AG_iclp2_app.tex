\NeedsTeXFormat{LaTeX2e}
\documentclass{tlp}
\usepackage{times}
\usepackage{helvet}
\usepackage{courier}
\usepackage{amsmath}
\usepackage{amssymb}
\usepackage{url}

\def\G{\Gamma}
\def\D{\Delta}
\def\seq{\Rightarrow}
\def\o{\overline}
\def\bp{\textbf{p}}
\def\ar{\leftarrow}
\def\rar{\rightarrow}
\def\lrar{\leftrightarrow}
\def\beq{\begin{equation}}
\def\eeq#1{\label{#1}\end{equation}}
\def\ba{\begin{array}}
\def\ea{\end{array}}
\def\i#1{\hbox{\it #1\/}}
\def\is#1{{\hbox{\scriptsize {\it #1\/}}}}
\def\C{{\cal C}}
\def\no{\i{not}}
\def\iif{\hbox{\bf if}}
\def\vars{\hbox{\rm Vars}}
\def\gvars{\hbox{\rm GVars}}
\def\lvars{\hbox{\rm LVars}}
\def\dom{\hbox{\rm Dom}}
\def\after{\hbox{\bf after}}
\def\ifcons{\hbox{\bf if}\;\!\hbox{\bf cons}}
\def\causes{\hbox{\bf causes}}
\def\inertial{\hbox{\bf inertial}}
\def\default{\hbox{\bf default}}
\def\constr{\hbox{\bf constraint}}
\def\imp{\hbox{\bf impossible}}
\def\nonex{\hbox{\bf nonexecutable}}
\def\pn{\hbox{\it PN}}
\def\ps{\hbox{\it PS}}
\def\mv{\hbox{\it MV}}
\def\choice#1{\i{Choice}(#1)}
\def\s{\hbox{\it SL}}
\def\d{\hbox{\it DL}}
\def\sm{\hbox{SM}}
\def\gr{\hbox{Gr}}
\def\r#1#2{\frac{\textstyle #1}{\textstyle #2}}
\hyphenation{lif-schitz}

\newtheorem{theorem}{Theorem}
\newtheorem{lemma}{Lemma}
\newtheorem{proposition}{Proposition}


\title[Online appendix]{{\large\textnormal{Online appendix for the paper}}   \\
Abstract Gringo
\\
{\large\textnormal{published in Theory and Practice of Logic Programming}}
}

\author[Gebser, Harrison, Kaminski, Lifschitz, and Schaub]
{Martin Gebser, Amelia Harrison, Roland Kaminski,
Vladimir Lifschitz and Torsten Schaub}


\begin{document}

\appendix

\title[Online appendix]{{\large\textnormal{Online appendix for the paper}}   \\
Abstract Gringo
\\
{\large\textnormal{published in Theory and Practice of Logic Programming}}
}
\author[Gebser, Harrison, Kaminski, Lifschitz, and Schaub]
{MARTIN GEBSER\\
Aalto University, HIIT, Finland\\
University of Potsdam, Germany\\
gebser@cs.uni-potsdam.de\\
\and
AMELIA HARRISON\\
Univeristy of Texas at Austin, USA\\
ameliaj@cs.utexas.edu\\
\and
ROLAND KAMINSKI\\
University of Potsdam, Germany\\
kaminski@cs.uni-potsdam.de\\
\and
VLADIMIR LIFSCHITZ\\
Univeristy of Texas at Austin, USA\\
vl@cs.utexas.edu\\
\and
TORSTEN SCHAUB\\
%\thanks{Affiliated with Simon Fraser
%University, Canada, and IIIS
%Griffith University, Australia.}\\
University of Potsdam, Germany\\
INRIA Rennes, France\\
torsten@cs.uni-potsdam.de\\
}

%\submitted {1 January 2003}
%\revised {1 January 2003}
%\accepted{1 January 2003}

\makeatletter
{%
  \newpage
  \begingroup
    \newpage
    \global\@topnum\z@
    \titlefntrue
    \def\thefootnote{\@fnsymbol\c@footnote}%
    \def\@makefnmark{\hbox{$\@thefnmark$}}%
    \thispagestyle{titlepage}\@thanks
  \endgroup
  \vspace*{-15\p@}%
  {\centering \sloppy
   \pe@rl{#1}%
   {\normalfont\LARGE\itshape \@title\par}%
   \vskip 16\p@
   {\normalfont\normalsize\rmfamily
    \let\authorbreak\auth@rbreak
    \let\and\@nd
    \b@at
      \@author
    \end{author@tabular}%
   \par}%
%   \ifprodtf
     %\vskip 10pt%
     %{{\affilsize\it submitted \@submitted; revised \@revised; accepted \@accepted}}\par
%   \fi
  }%
 % \vskip 18\p@ \@plus 2\p@ \@minus 1\p@
}
\makeatother

\section*{Proofs of Theorems~1--3}
\label{app:proofs}

\begin{proof*}[Proof of Theorem~1]
We will show that for any term $t$, the set of conjunctive terms of
$\tau_t E$ is the union
of the sets of conjunctive terms of $\tau_t E_{\leq}$ and $\tau_t E_{\geq}$.
% Consider the set $A$ defined in Section \ref{sem:ag}.
For any subset $\Delta$ of~$A$,
\begin{center}
\begin{tabular}{l  l}
    & (22) is a conjunctive term of $\tau_t E$ \\
iff & $\Delta$ does not justify $E$ with respect to $t$ \\
iff & $\widehat{\alpha}[\Delta] \not = t$ \\
iff & $\widehat{\alpha}[\Delta] < t$ or $\widehat{\alpha}[\Delta] > t$ \\
iff & $\Delta$ does not justify $E_{\ge}$ with respect to $t$ or $\Delta$ does not justify
$E_{\le}$ with respect to $t$\\
iff & (22) is a conjunctive term of $\tau_t E_{\ge}$ or of
$\tau_t E_{\le}$\\
iff & (22) is a conjunctive term of $\tau_t E_{\le} \land \tau_t E_{\ge}.$
\end{tabular}
\end{center}
\end{proof*}

\newpage

\begin{proof*}[Proof of Theorem~2]
Since $[\o m]$ is the singleton set $\{\o m\}$,
$\tau E$ is $\tau_{\o m} E$.
Since~$E$ is monotone,
the antecedent of~(22) can be dropped
(Section~5.1), so that $\tau_{\o m} E$ is strongly equivalent to
%\beq
\[
  \bigwedge_{ \Delta \subseteq A \atop \lvert [\Delta] \rvert < m}
  \bigvee_{(i,{\bf r}) \in A\setminus\Delta}
  \tau_{\lor}(({\bf L}_i)^{{\bf x}_i}_{\bf r}).
\tag{25}
%\eeq
\label{mono_geq}
\]
To derive (25) from (23) in HT\/$^\infty$,
assume (23). We will reason by cases, with one case
corresponding to each disjunctive term
%\beq
\[
  \bigwedge_{(i,{\bf r}) \in \Delta}
  \tau_{\lor}(({\bf L}_i)^{{\bf x}_i}_{\bf r})
\tag{26}
%\eeq
\label{dterm}
\]
of (23).
Let $\Delta'$ be a subset of $A$ such that
$\lvert [\Delta']\rvert < m$.
We will show that the conjunctive term
of (25) corresponding to $\Delta'$ can be derived from
(26).  Since
%\beq
\[
\lvert [\Delta'] \rvert < m = \lvert [\Delta]\rvert,
\tag{27}
%\eeq
\label{ineq}
\]
there exists a pair $(i, {\bf r})$ that is an element of $\Delta$ but
not an element of $\Delta'$. Indeed, if \hbox{$\Delta \subseteq \Delta'$} then
$[\Delta] \subseteq [\Delta']$, which contradicts (27).
Since $(i, {\bf r}) \in \Delta$, from~(26)
we can derive $\tau_{\lor}(({\bf L}_i)_{\bf r}^{{\bf x}_i})$. Since
$(i, {\bf r}) \in A \setminus \Delta'$, we can further derive
$$
 \bigvee_{(i,{\bf r}) \in A\setminus\Delta'}
 \tau_{\lor}(({\bf L}_i)^{{\bf x}_i}_{\bf r}).
$$
It follows that each conjunctive term of (25)
can be derived from (26).

We will prove by induction on $m$ that
(23) can be derived from (25) in HT\/$^\infty$.
Base case: when $m = 0$ the disjunctive term of (23)
corresponding to the empty $\Delta$ is $\top$.
Inductive step: assume that (23) can be derived from
(25), and assume
%\beq
\[
  \bigwedge_{\Delta \subseteq A \atop \lvert [\Delta]\rvert < m + 1}
  \bigvee_{(i,{\bf r}) \in A\setminus\Delta}
  \tau_{\lor}(({\bf L}_i)^{{\bf x}_i}_{\bf r}).
%\eeq
\tag{28}
\label{monok}
\]
From (28) we can derive (25), and consequently
(23).
Now we reason by cases, with one case corresponding to each disjunctive
term of (23).
Assume
%\beq
\[
  \bigwedge_{(i,{\bf r}) \in \Sigma}
  \tau_{\lor}(({\bf L}_i)^{{\bf x}_i}_{\bf r})
%\eeq
\tag{29}
\label{dterm2}
\]
where $\Sigma$ is a subset of $A$ such that $\lvert [\Sigma]\rvert = m$.
Consider the set
$$\Sigma' = \{(i, {\bf r}) : [({\bf t}_i)^{{\bf x}_i}_{\bf r}]
\subseteq [\Sigma]\}.$$
By the definition of $[\Sigma]$, for any $(i, {\bf r}) \in \Sigma$,
$[({\bf t}_i)^{{\bf x}_i}_{\bf r}] \subseteq [\Sigma]$. So $\Sigma \subseteq
\Sigma'$. It follows that $[\Sigma] \subseteq [\Sigma']$.
On the other hand,
$$[\Sigma'] =
\bigcup_{(i, {\bf r}) \in \Sigma'} [({\bf t}_i)^{{\bf x}_i}_{\bf r}] =
\bigcup_{(i, {\bf r})\;:\;
[({\bf t}_i)^{{\bf x}_i}_{\bf r}] \subseteq [\Sigma]}
[({\bf t}_i)^{{\bf x}_i}_{{\bf r}}] \subseteq [\Sigma].$$
Consequently $[\Sigma] = [\Sigma']$, and
$\lvert [\Sigma'] \rvert = \lvert [\Sigma] \rvert = m.$
From (28),
%\beq
\[
  \bigvee_{(i,{\bf r}) \in A\setminus\Sigma'}
  \tau_{\lor}(({\bf L}_i)^{{\bf x}_i}_{\bf r}).
%\eeq
\tag{30}
\label{monok_res}
\]
Again, we reason by cases, with one case corresponding to each disjunctive
term of (30). Assume $\tau_{\lor}(({\bf L}_j)^
{{\bf x}_j}_{\bf s})$, where $(j, {\bf s}) \in A \setminus \Sigma'$.
Combining assumption (29) and $\tau_{\lor}(({\bf L}_j)^
{{\bf x}_j}_{\bf s})$, we derive
%\beq
\[
  \bigwedge_{(i, {\bf r}) \in \Sigma \cup \{(j,{\bf s})\}}
  \tau_{\lor}(({\bf L}_i)^{{\bf x}_i}_{\bf r}).
%\eeq
\tag{31}
\label{combin}
\]
Consider the set $[\Sigma \cup \{(j, {\bf s})\}]$, that is,
%\beq
\[
[\Sigma] \cup [({\bf t}_j)^{{\bf x}_j}_{\bf s}].
%\eeq
\tag{32}
\label{31}
\]
Recall that the cardinality
of $[\Sigma]$ is $m$. Since ${\bf t}_j$ is interval-free,
the cardinality of $[({\bf t}_j)^{{\bf x}_j}_{\bf s}]$ is at most $1$.
Furthermore,
since $(j, {\bf s}) \not \in \Sigma'$ it follows that
$$[({\bf t}_j)^{{\bf x}_j}_{\bf s}] \not \subseteq [\Sigma],$$
so that $[({\bf t}_j)^{{\bf x}_j}_{\bf s}]$ is nonempty.
Consequently,
the set is a singleton,
and therefore $[\Sigma]$ is disjoint from it.
It follows that the cardinality of (\ref{31}) is $m+1$.
%\hbox{$\lvert [\Sigma \cup \{(j, {\bf s})\}] \rvert = m+1$}.
So from (\ref{combin}) we can derive
$$
  \bigvee_{\Delta\subseteq A \atop \lvert [\Delta] \rvert = m+1}
  \bigwedge_{(i,{\bf r}) \in \Delta}
  \tau_{\lor}(({\bf L}_i)^{{\bf x}_i}_{\bf r}).
$$
\end{proof*}

\begin{proof*}[Proof of Theorem~3]
%\noindent {\bf Proof of Theorem~\ref{lem:leq}}$\;$
Since $[\o m]$ is the singleton set $\{\o m\}$,
$\tau E$ is $\tau_{\o m} E$.
Since the
consequent of~(22) can be replaced in this case by~$\bot$,
$\tau_{\o m} E$ is strongly equivalent to
%\beq
\[
  \bigwedge_{\Delta \subseteq A \atop \lvert [\Delta] \rvert > m}
  \neg \bigwedge_{(i, {\bf r}) \in\Delta}
  \tau_{\lor}(({\bf L}_i)^{{\bf x}_i}_{\bf r}).
%\eeq
\tag{33}
\label{trans}
\]
Every conjunctive term of (24) is a conjunctive term of
(\ref{trans}). To derive (\ref{trans}) from (24), consider a
set $\Delta$ such that $\lvert [\Delta] \rvert > m$.
Let $f(i,{\bf r})$ stand for the set
$[({\bf t}_i)^{{\bf x}_i}_{\bf r}]$. Since each ${\bf t}_i$ is
interval-free, this set is either empty or a singleton.
%So $[\Delta]$ is the union of the singletons $f(i,{\bf r})$
%for all $(i,{\bf r})$ from $\Delta$.
Let ${\bf s}_1,\dots,{\bf s}_{m+1}$
be $m+1$ distinct elements of $[\Delta]$.
Choose elements $(i_1,{\bf r}_1),\dots,(i_{m+1},{\bf r}_{m+1})$ of
$\Delta$ such that each $s_k$ belongs to $f(i_k,{\bf r}_k)$, and let
$\Delta'$ be $\{(i_1,{\bf r}_1),\dots,(i_{m+1},{\bf r}_{m+1})\}$.
The cardinality of $[\Delta']$ is at least $m+1$, because this set
includes ${\bf s}_1,\dots,{\bf s}_{m+1}$.  On the other hand, it is at
most $m+1$, because this set is the union of $m+1$ sets of cardinality at
most $1$.
Consequently, $\lvert [\Delta']\rvert = m+1$.  From (24)
we can conclude in HT\/$^\infty$ that
%\beq
\[
  \neg \bigwedge_{(i, {\bf r}) \in\Delta'} \tau_{\lor}
  (({\bf L}_i)^{{\bf x}_i}_{\bf r}).
%\eeq
\tag{34}
\label{cterm}
\]
Then the conjunctive term
$$\neg \bigwedge_{(i, {\bf r}) \in\Delta}
  \tau_{\lor}(({\bf L}_i)^{{\bf x}_i}_{\bf r})$$
of~(\ref{trans}) follows, because $\Delta' \subseteq \Delta$.\end{proof*}

\section*{Correctness of the $n$-Queens Program} \label{sec:nqueens}

In this section, we prove the correctness of the program $K$,
consisting of rules $R_1,\dots,R_7$ (Sections~2.3 and~3).

The $n$-queens problem involves placing $n$ queens on an $n \times n$
chess board such that no two queens threaten each other. We will represent
squares by pairs of integers $(i, j)$ where
\hbox{$1 \leq i,j \leq n$}. Two squares $(i_1, j_1)$ and $(i_2, j_2)$ are said
to be in the same row if $i_1 = i_2$; in the same column
if $j_1 = j_2$; and in the same diagonal if $\lvert i_1 - i_2 \rvert =
\lvert j_1 - j_2 \rvert$.
A set $Q$ of $n$ squares is a {\sl solution} to the $n$-queens problem if
no two elements of $Q$ are in the same row,
in the same column,
or in the same diagonal.

For any stable model $I$ of $K$, by $Q_I$ we denote the set of
pairs $(i, j)$ such that $q(\o i, \o j) \in I$.

\medskip

\noindent\emph{Theorem 4}\\
For each stable model $I$ of $K$, $Q_I$ is a
solution to the $n$-queens problem.  Furthermore, for each solution $Q$ to
the $n$-queens problem there is exactly one stable model $I$ of $K$ such that
$Q_I = Q$.

\medskip

\subsection*{Review: Supported Models and Constraints}

We start by reviewing two familiar facts that will be
useful in proving Theorem~4.

An {\sl infinitary program} is a conjunction of (possibly infinitely
many) infinitary formulas of the form $G \rar A$, where $A$ is an atom.
We say that an interpretation $I$ is
{\sl supported} by an infinitary program $\Pi$ if each atom $A$ from $I$ is
the consequent of a conjunctive term~$G\rar A$ of $\Pi$ such that $I\models G$.
\citeANP{lif13a} \citeyear{lif13a} give a condition, ``tightness on an
interpretation,''  under which the stable models of
an infinitary program are identical to its supported models.
Proposition~\ref{thm:etight} below
gives a simpler condition of this kind that is sufficient for our purposes.

We say that an atom $A$ {\sl occurs nonnegated in a formula $F$} if
\begin{itemize}
\item $F$ is $A$, or
\item $F$ is of the form $\mathcal{H}^\land$ or $\mathcal{H}^\lor$ and $A$ occurs nonnegated in at least one element of $\mathcal{H}$, or
\item $F$ is of the form $G \rar H$, where $H$ is different from $\bot$, and
$A$ occurs nonnegated in $G$ or in $H$.
\end{itemize}
It is clear, for instance, that no atom occurs nonnegated in a formula of the
form $\neg F$.

The {\sl positive dependency graph} of an infinitary program $\Pi$ is the
directed graph containing a vertex for each atom occuring in $\Pi$, and an
edge from $A$ to $B$ for every conjunctive term $G \rar A$ of $\Pi$ and every
atom $B$ that occurs nonnegated in $G$.
We say that an infinitary program $\Pi$ is {\sl extratight}
if the positive dependency graph of $\Pi$ contains no infinite paths.

The following fact is immediate from \cite[Lemma 2]{lif13a}.

\begin{proposition}\label{thm:etight}
For any model $I$ of an extratight infinitary program $\Pi$, $I$ is
stable iff $I$ is supported by~$\Pi$.
\end{proposition}

A {\sl constraint} is an infinitary formula of the form $\neg F$
(which is shorthand for $F\rar\bot$). The
following theorem is a straightforward generalization of
Proposition 4  from \cite{fer05e}.

\begin{proposition}\label{thm:cons}
Let $\mathcal{H}_1$  be a set of infinitary formulas and $\mathcal{H}_2$ be
a set of constraints. A set $I$ of
atoms  is a stable model of $\mathcal{H}_1 \cup \mathcal{H}_2$ iff $I$ is a
stable model of $\mathcal{H}_1$ and satisfies all formulas in~$\mathcal{H}_2$.
\end{proposition}

\begin{proof*}
%\noindent {\bf Proof} $\;$
{\em Case 1:} Every formula in $\mathcal{H}_1 \cup \mathcal{H}_2$ is satisfied
by $I$. For each formula $\neg F$ in $\mathcal{H}_2$, $I$ does not satisfy $F$.
So the reduct of each formula in $\mathcal{H}_2$ w.r.t. $I$ is $\neg \bot$.
It follows that the set of reducts of all formulas in
$\mathcal{H}_1 \cup \mathcal{H}_2$
is satisfied by the same interpretations as the set of reducts of all
formulas in $\mathcal{H}_1$.  Consequently, $I$ is minimal among the sets
satisfying the reducts of all formulas from
$\mathcal{H}_1 \cup \mathcal{H}_2$ iff it is minimal among the sets
satisfying the reducts of all formulas from $\mathcal{H}_1$. {\em Case~2: }
Some formula $F$ in $\mathcal{H}_1 \cup \mathcal{H}_2$ is not satisfied by $I$.
Then $I$ is not a stable model of $\mathcal{H}_1 \cup \mathcal{H}_2$.
If $F \in \mathcal{H}_1$ then $I$ is not a stable model of $\mathcal{H}_1$.
Otherwise, it is not true that $I$ satisfies all formulas in $\mathcal{H}_2$.
\end{proof*}

\subsection*{Proof of Theorem 4}

To simplify notation, we will identify each set $Q$ of squares with the set of
atoms $q(\o i, \o j)$ where \hbox{$(i, j) \in Q$}. By $D_n$ we denote the set
of atoms of the forms $\i{d1}(\o i, \o j, \o{i-j+n})$ and $\i{d2}(\o i, \o j,
\o{i+j-1})$ for all $i,j$ from $\{1,\dots,n\}$. Recall that the rules
of the program $K$ are denoted by $R_1,\dots,R_7$.

\begin{lemma}\label{lem:sup}
A set of atoms is a stable model of
%\beq
\[
\tau R_1 \cup \tau R_4 \cup \tau R_5
%\eeq
\tag{35}
\label{supset}
\]
iff it is of the form $Q \cup D_n$ where $Q$ is a set of squares.
\end{lemma}

\begin{proof*}
%\noindent {\bf Proof} $\;$
We can turn (\ref{supset}) into a strongly
equivalent infinitary program as follows. The result of applying $\tau$ to
$R_1$ is (21).
%the conjunction of the formulas
%$$
%\beq
%  q(\o i, \o j) \lor \neg q(\o i, \o j)
%\eeq{tauR_1}
%$$
%$(1 \leq i,j \leq n)$,
%which are strongly equivalent to
Each conjunctive term in this formula is strongly equivalent to
%\beq
\[
\neg \neg q(\o i, \o j) \rar q(\o i, \o j).
%\eeq
\tag{36}
\label{etightR_1}
\]
The set $\tau R_4$ is strongly equivalent to
the set of formulas
%\beq
\[
  \top \rar d1(\o i,\o j,\o{i-j+n})
%\eeq
\tag{37}
\label{r4}
\]
($1 \leq i,j \leq n$).
(We take into account that $\tau(\o i = \o 1..\o n)$ is equivalent to
$\top$ if $1 \leq i \leq n$ and to~$\bot$ otherwise, and similarly for $j$.)
Similarly, $\tau R_5$ is strongly equivalent to the set of formulas
%\beq
\[
  \top \rar d2(\o i,\o j,\o{i+j-1})
%\eeq
\tag{38}
\label{r5}
\]
$(1 \leq i,j \leq n)$.
Consequently, (\ref{supset}) is strongly equivalent to the conjunction $H$ of
formulas (\ref{etightR_1})--(\ref{r5}).
It is easy to check that $H$ is an extratight infinitary program, so that
by Proposition~\ref{thm:etight} its stable models are identical to its
supported models.
A set $I$ of atoms is a model of $H$ iff $D_n \subseteq I$. Furthermore,
$I$ is supported iff every element of $I$ has the form $q(\o i, \o j)$ or
is an element of $D_n$. Consequently, supported models of $H$ are sets of the
form $Q \cup D_n$ where $Q$ is a set of squares.
\end{proof*}

\begin{lemma}\label{lem:main}
A set $I$ of atoms is a stable model of $\tau K$ iff
it has the form $Q \cup D_n$, where $Q$ is a solution to the $n$-queens
problem.
\end{lemma}

%\noindent{\bf Proof} \;\;
\begin{proof*}
Let $\mathcal{H}_1$ be (\ref{supset}) and $\mathcal{H}_2$ be
$$
\tau R_2 \cup \tau R_3 \cup \tau R_6 \cup \tau R_7.
$$
All formulas in $\mathcal{H}_2$ are constraints. Consequently,
by Proposition~\ref{thm:cons}, $I$ is a stable
model of $\tau K$ iff it is a stable model of $\mathcal{H}_1$ and satisfies all formulas in $\mathcal{H}_2$.
By Lemma~\ref{lem:sup}, $I$ is a stable model of $\mathcal{H}_1$ iff it is
of the form $Q \cup D_n$, where $Q$ is a set of squares. It remains to show
that a set $I$ of the form $Q \cup D_n$ satisfies all formulas in
$\mathcal{H}_2$ iff $Q$ is a solution to the $n$-queens problem.
Specifically, we will show that for any set $I$ of the form $Q \cup D_n$
\begin{itemize}
\item[(i)] $I$ satisfies $\tau R_2$ iff for all $i \in \{1,\dots,n\}$, $I$
contains exactly one atom of the form $q(\o i, \o j)$;
\item[(ii)] $I$ satisfies $\tau R_3$ iff~for all $j \in \{1,\dots,n\}$, $I$
contains exactly one atom of the form $q(\o i, \o j)$;
\item[(iii)] $I$ satisfies $\tau R_6 \cup \tau R_7$ iff no two squares in $I$
are in the same diagonal.
\end{itemize}

To prove (i), note first that $\tau R_2$ is equivalent to  the set of formulas
$$
\neg \neg \left (\tau(\i{count}\{Y: q(\o i,Y)\} = \o 1) \right )
$$
$(1 \leq i \leq n)$. Let $E$ be the aggregate atom above. Since $[\o 1]$ is a
singleton set, $\tau E$ is the same as $\tau_{\o 1} E$. By Theorem~1, this set is strongly
equivalent to the set of formulas
%\beq
\[
\neg \neg \left (\tau_{\o 1}(\i{count}\{Y:
  q(\o i,Y)\} \leq \o 1) \land \tau_{\o 1}(\i{count}\{Y: q(\o i,Y)\} \geq \o 1)
  \right ).
%\eeq
\tag{39}
\label{r2expand}
\]
Again note that the result of applying $\tau$ to first aggregate atom in \eqref{r2expand}
is the same as the result of applying $\tau_{\o 1}$.
Then by Theorem~3 and the comment at the end of Section~5.3,
$\tau$ applied to this
aggregate atom is strongly equivalent to
$$
\bigwedge_{\Delta \subseteq A \atop \lvert \Delta \rvert = 2} \neg
  \bigwedge_{(1, r) \in \Delta} q(\o i, r).
$$
This formula can be written as
$$
\bigwedge_{\Sigma \subseteq P \atop \lvert \Sigma \rvert = 2} \neg
  \bigwedge_{r \in \Sigma} q(\o i, r),
$$
where $P$ is the set of precomputed terms.
It is easy to see that $I$ satisfies this formula iff it contains at most
one atom of the form  $q(\o i,r)$. On the other hand, by Theorem~2,
the result of applying $\tau$ to the second aggregate atom in (\ref{r2expand})
is strongly equivalent to
$$
  \bigvee_{\Delta \subseteq A \atop \lvert \Delta \rvert = 1}
  \bigwedge_{(1, r) \in\Delta} q(\o i, r).
$$
Similar reasoning shows that $I$ satisfies this formula iff it
contains at least
one atom of the form  $q(\o i,r)$. Since $I = Q \cup D_n$,
$r$ in this atom is one of $\o 1,\dots, \o n$.

Claim (ii) is proved in a similar way.

To prove (iii), note first that two squares
$(\o{i_1}, \o{j_1}), (\o{i_2}, \o{j_2})$ are in the same diagonal iff
there exists a $k \in \{1, \dots, 2n-1\}$ such that
%\beq
\[
d1(\o i_1, \o j_1, \o k), d1(\o i_2, \o j_2, \o k) \in D_n
%\eeq
\tag{40}
\label{c1}
\]
or
%\beq
\[
d2(\o i_1, \o j_1, \o k), d2(\o i_2, \o j_2, \o k) \in D_n.
%\eeq
\tag{41}
\label{c2}
\]
We will show that a set $I$ of the form $Q \cup D_n$ does not satisfy
$\tau R_6$ iff
there exists a $k$ such that~(\ref{c1}) holds for two distinct elements
$q(\o{i_1}, \o{j_1}), q(\o{i_2}, \o{j_2}) \in Q$, and that it does not
satisfy
$\tau R_7$ iff there exists a $k$ such that (\ref{c2}) holds for such two
elements.  The result of applying $\tau$ to $R_6$ is strongly
equivalent to the set of formulas
%\beq
\[
  \neg \tau(2 \leq \i{count}\{\o 0, q(X,Y) : q(X,Y), d1(X,Y,\o k)\})
%\eeq
\tag{42}
\label{tauR_6}
\]
$(1 \leq k \leq 2n-1)$. Formula (\ref{tauR_6}) is identical to
$$
\neg \tau ( \i{count}\{X,Y : q(X,Y), d1(X,Y,\o k)\} \geq 2 ).
$$
In view of Theorem~2, it follows that it is strongly equivalent
to
$$
\neg \bigvee_{\Delta \subseteq A \atop \lvert \Delta
 \rvert = 2} \bigwedge_{(1, (r,s)) \in \Delta} (q(r, s) \land d1(r, s,  \o k))
$$
$(1 \leq k \leq 2n-1)$. This formula can be written as
%\beq
\[
 \neg \bigvee_{\Sigma \subseteq P \times P \atop \lvert \Sigma
 \rvert = 2} \bigwedge_{(r,s) \in \Sigma} (q(r, s) \land d1(r, s,  \o k)).
%\eeq
\tag{43}
\label{r6simp}\]
For any set $Q$ of squares,
\begin{center}
\begin{tabular}{l  l}
    & $Q \cup D_n$ does not satisfy (\ref{r6simp}) \\
iff & there exist two distinct pairs $(r_1, s_1), (r_2, s_2)$ from
    $P \times P$ such that\\
    & $q(r_1,s_1),q(r_2, s_2) \in Q$ and $d1(r_1,s_1,\o k),
    d1(r_2, s_2, \o k) \in D_n$\\
iff & there exist two distinct squares $(\o i_1, \o j_1), (\o i_2, \o j_2) \in
    Q$ such that (\ref{c1}) holds.
\end{tabular}
\end{center}

The claim about (\ref{c2}) is proved in a similar way.
\end{proof*}
%\medskip

Theorem~4 is immediate from the lemma.

% \bibliographystyle{acmtrans}
% \bibliography{bib}

\begin{thebibliography}{}

\bibitem[\protect\citeauthoryear{Ferraris and Lifschitz}{Ferraris and
  Lifschitz}{2005}]{fer05e}
{\sc Ferraris, P.} {\sc and} {\sc Lifschitz, V.} 2005.
\newblock Mathematical foundations of answer set programming.
\newblock In {\em We Will Show Them! Essays in Honour of Dov Gabbay}. King's
  College Publications, 615--664.

\bibitem[\protect\citeauthoryear{Lifschitz and Yang}{Lifschitz and
  Yang}{2013}]{lif13a}
{\sc Lifschitz, V.} {\sc and} {\sc Yang, F.} 2013.
\newblock Lloyd-{T}opor completion and general stable models.
\newblock {\em Theory and Practice of Logic Programming\/}~{\em 13,\/}~4--5.

\end{thebibliography}


\end{document}



